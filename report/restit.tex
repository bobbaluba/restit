\documentclass{ml}

\begin{document}
\maketitle{Rest-It}{Sun, 09 Jun 2013 23:59}{Mareike Picklum}{z3446393}{Johan Klokkhammer Helsing}{z3432903}

%------------------------------------------------------------------------------

\section{Introduction}
\label{intro}
The project \textit{Rest-It} is an implementation of a reinforcement learner for the popular puzzle game Tetris. 
The core idea of the game is, to place Tetronimoes (pieces consisting of connected cells) in the way that the complete as many lines as possible in a 10 columns by 20 rows playing field.
Each completed line is removed from the field and the cells above are moved one cell down.
Removing a line adds one to the players score.
Tetris is a benchmark problem for testing machine learning algorithms, as it uses comparably simple rules although it requires complex strategies to perform well. 

In \nameref{relatedwork} we will give a short overview about existing reinforcement learning approaches for Tetris and give information about what our project is based on and influenced by.
We will then describe the algorithms we implemented \nameref{method} and present  the results for the different approaches in \nameref{results}.
We will give a short summary and wrap up in \nameref{conclusions}.

%------------------------------------------------------------------------------

\section{Related work}
\label{relatedwork}
There are different approaches for tetris learners. 
Csaba Szepesvári \cite{szepesvari2010algorithms} describes different reinforcement learning algorithms and their core ideas and discusses their theoretical  limitations. \\
Szita \&  L{\"o}rincz \cite{szita2006learning} apply noise to improve the cross-entropy method by preventing it from early convergence and achieve a policy that outperforms previous algorithms by almost two orders of magnitude.\\
Donald Carr introduces reinforcement learning for Tetris and introduces the Formal Tetris Specification by Fahey \cite{faheytetris}.\\
Gross \& Schwenker \cite{gross2008learningto} investigate the idea of temporal difference learning applied to the Tetris problem and train agent using an $\varepsilon$-greedy policy. 

Our implementation is based on an open source Qt Project\footnote{\href{https://qt-project.org/doc/qt-4.8/widgets-tetrix.html}{Qt Project}} implementation. \\
The core idea for the reinforcement learning algorithm we use is described by Zucker \& Maas \cite{zucker2009learning}, which will be investigated in more detail in \autoref{method}.

%------------------------------------------------------------------------------
\\
\section{Method}
\label{method}

We use 

\lipsum[9-12]

%------------------------------------------------------------------------------

\section{Results}
\label{results}

\lipsum[1-2] 

\begin{wrapfigure}{r}{0.5\textwidth}
\vspace{-10pt}
  \begin{center}
    \includegraphics[width=0.5\textwidth]{img/woooo1.png}
  \end{center}
  \vspace{-10pt}
  \caption{\small{WOOOOOO!}}
  \label{wooo}
  \vspace{-10pt}
\end{wrapfigure}

\lipsum[3-4]


\begin{wrapfigure}{l}{0.5\textwidth}
\vspace{-10pt}
  \begin{center}
    \includegraphics[width=0.5\textwidth]{img/superhappyawesome.png}
  \end{center}
  \vspace{-10pt}
  \caption{\small{SUPERHAPPYAWESOME}}
  \label{superhappyawesome}
  \vspace{-10pt}
\end{wrapfigure}

\lipsum[5-6]

%------------------------------------------------------------------------------

\section{Conclusions}
\label{conclusions}

\lipsum[7-9]

%------------------------------------------------------------------------------

%%%
% Sources
\newpage
\bibliographystyle{alphadin}
\bibliography{sources/sources}
\end{document}
